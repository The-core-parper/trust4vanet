\documentclass[compsoc, conference, letterpaper, 10pt, times]{IEEEtran}
\usepackage{graphicx}
\usepackage{amssymb}
\usepackage{amsmath}
\usepackage{amsthm}
% \usepackage{amsart}
\usepackage{bussproofs}
\usepackage{url}
\usepackage{listings}


\newtheorem{definition}{Definition}
%\newtheorem{proposition}{Proposition}
%\newtheorem{thesis}{Thesis}
\newtheorem{theorem}{Theorem}
%\newtheorem{corollary}{Corollary}
%\newtheorem{lemma}{Lemma}


                       \lstset{ %
                       	%backgroundcolor=\color{white},   % choose the background color; you must add \usepackage{color} or \usepackage{xcolor}
                       	%basicstyle=\normalfont\ttfamily,        % the size of the fonts that are used for the code
                       	%basicstyle=\normalfont\footnotesize\ttfamily,        % the size of the fonts that are used for the code
                       	aboveskip={-.4cm},
                       	breakatwhitespace=false,         % sets if automatic breaks should only happen at whitespace
                       	%breaklines=true,                 % sets automatic line breaking
                       	%captionpos=b,                    % sets the caption-position to bottom
                       	commentstyle=\color{Plum},    % comment style
                       	morecomment=[l]{--},
                       	deletekeywords={exp,eq,values,return,room},            % if you want to delete keywords from the given language
                       	escapeinside={(*}{*)},          % if you want to add LaTeX within your code
                       	%extendedchars=true,              % lets you use non-ASCII characters; for 8-bits encodings only, does not work with UTF-8
                       	%frame=single,                    % adds a frame around the code
                       	%keepspaces=true,                 % keeps spaces in text, useful for keeping indentation of code (possibly needs columns=flexible)
                       	keywordstyle=\color{blue},       % keyword style
                       	language=Java,                 % the language of the code
                       	morekeywords={define,let,cond,if,struct},            % if you want to add more keywords to the set
                      %	numbers=left,                    % where to put the line-numbers; possible values are (none, left, right)
                       	%numbersep=5pt,                   % how far the line-numbers are from the code
                       	numberstyle=\tiny\color{black}, % the style that is used for the line-numbers
                       	%rulecolor=\color{black},         % if not set, the frame-color may be changed on line-breaks within not-black text (e.g. comments (green here))
                       	showspaces=false,                % show spaces everywhere adding particular underscores; it overrides 'showstringspaces'
                       	showstringspaces=false,          % underline spaces within strings only
                       	showtabs=false,                  % show tabs within strings adding particular underscores
                       	%stepnumber=2,                    % the step between two line-numbers. If it's 1, each line will be numbered
                       	stringstyle=\color{OliveGreen},     % string literal style
                       	tabsize=2,                       % sets default tabsize to 2 spaces
                       	basicstyle=\ttfamily,columns=flexible,
                       	title=\lstname                   % show the filename of files included with \lstinputlisting; also try caption instead of title
                       	%numbers=left,
                       	%basicstyle=\scriptsize,
                       	%mathescape
                       }

\begin{document}
%opening
\title{A Proof-theoretic Trust and Reputation Model for VANET}
%\author{Giuseppe Primiero}
%\institute{Department of Computer Science\\Middlesex University London\\United Kingdom}


\author{\IEEEauthorblockN{Giuseppe Primiero, Franco Raimondi, Taolue Chen, Rajagopal Nagarajan}
	\IEEEauthorblockA{Department of Computer Science,\\
		Middlesex University London,\\
		United Kingdom\\
		Email: G.Primiero|F.Raimondi|T.Chen|R.Nagarajan@mdx.ac.uk}}


\maketitle

\begin{abstract}
Vehicle Ad Hoc Networks (VANETs) are becoming an important part of intelligent transportation systems. In this context, security requirements need to rely on a combination of vehicles' reputation and trust relations over the messaging infrastructure, in order to maintain a dynamic and safe behaviour evaluation. Formal correctness, resolution of contradictions, as well as proven safety of transitive operations in the presence of reputation and trust within the infrastructure, remain mostly unexplored issues with potentially disastrous effects. In this paper we provide a proof-theoretic interpretation of a reputation and trust model for VANET, which allows for formal verification through a translation into the Coq proof assistant, and which guarantees consistency of messaging protocols and security of transitive transmissions.
\end{abstract}


\section{Introduction}\label{sec:intro}

Vehicle Ad Hoc Networks (VANETs) consist of vehicles and roadside units networks created to enhance transportation systems through vehicle-to-vehicle (V2V) and vehicle-to-infrastructure (V2I) communications. Due to their distributed and dynamic nature, such networks are open to several types of threats, including false message propagation. Trust and reputation are among the  most used concepts to ensure integrity, reliability and safety of services. Several methods have been implemented in VANETs to manage trust, see \cite{Soleymani2015} for a recent overview. Trust models in VANETs differ in accordance to the main object of the model: entity-centric \cite{5641621, GomezMarmol:2012:TTR:2160992.2161100}, data-centric \cite{conf/infocom/RayaPGH08, Lo2009} and combined \cite{Wei2012}. Among the models that combine trust and reputation, the work in~\cite{glenford} gives an analysis that accounts for reputation as a characteristic of message forwarding, as well as vehicles, drivers and other agents: here reputation is therefore based on a descriptive ontology of the model and is used to provide feedback in the system. An overview of the issues related to the trust in fixed and mobile ad hoc networks is given in \cite{DBLP:conf/vtc/WexBHLD08}, while other approaches for trustworthiness and reputation in ad hoc mobile networks are presented, for example, in \cite{DBLP:conf/um/FinnsonZTMC12, DBLP:journals/ijaisc/ChaurasiaTV15}.

In most of these models, the analysis relies on simulations. Yet, such
simulations cannot guarantee the absence of unpredictable and unsafe
behaviours. Since VANETs are meant to include safety and emergency messages,
more reliable methods are essential. The only method to produce exhaustive
safety control is through formal verification, but unfortunately none of the
current trust and reputation models seem to have focused on a formal
correctness requirement to ensure that the protocols are checkable. Formal
approaches to VANET include the work in \cite{DBLP:conf/vtc/KonurF11} for
the verification of a congestion control protocol using the model checker PRISM to investigate its correctness and effectiveness; privacy and authentication are verified using the AVISPA tool in~\cite{bouassida2011authentication}, while the TESLA authentication protocol is verified in~\cite{tesla-cpn} using Petri nets.  None of these (few) approaches focus explicitly on trust or reputation and they are all based on model checking. Other formal verification techniques like theorem proving seem to have been ignored so far. Moreover, an additional problem, i.e., ensuring that safety is preserved over transitive operations, remains unexplored.

The present paper provides a solution to both problems mentioned above. In Section \ref{sec:logic} we formulate a proof-theoretic translation of the trust and reputation model for VANET given in \cite{glenford} into an extension of the natural deduction calculus \texttt{(un)SecureND} from \cite{DBLP:conf/ifiptm/Primiero16}. The aim is, first of all, to show that the trust properties instantiated through our calculus faithfully reflect those in a VANET network; accordingly, we show how non-trustworthy interactions can be proven to be so through a proof-checking method. On a higher level, the model offered by \texttt{(un)SecureND} has been proven formally correct through its translation to a Coq library. As such, the present translation guarantees a similar property for the whole VANET model. Thanks to the structural properties of our calculus, we show how transitive message passing operations, in the form of instances of a cut rule, are guaranteed safe via applying a normalization result. In Section \ref{sec:opportunistic} we illustrate protocols for handshaking, recipient selection and message passing based on reputation. In Section \ref{sec:reputation} we give a reputation model based on an evaluation of feedback messages parametrized, in view of a temporal measure and a ranking of the relevant service characteristic of the message.


%\section{Related Work}\label{sec:related}


\section{\texttt{(un)SecureND}}\label{sec:logic}

$\mathtt{(un)SecureND}$ is a natural deduction calculus defining trust, mistrust and distrust protocols introduced in \cite{primiero_secureND} for the positive fragment,  and in \cite{DBLP:conf/ifiptm/Primiero16} for the negation complete extension. Here we provide  a slightly modified version adapted for a VANET network. In particular, the present version introduces: contexts as sets of sets; formulas with multiple indices to account for service and message numbers; ranking on service characteristics. We start with introducing the language of our logic:

\begin{definition}[Syntax of $\mathtt{(un)SecureND}$]\label{def:syntax} %The syntax  is defined by the following alphabet:
	%
	\begin{displaymath}
	\begin{array}{l}
	\mathcal{A}^{\prec}:= \{\mathcal{V, R}\} \quad 
	\mathcal{V}:= \{v_{1}\prec \cdots \prec v_{n}\}\\
	\mathcal{R}:= \{rsu_{1}\prec\cdots \prec rsu_{m}\}\quad \mathcal{S}:= \{S_1, \dots, S_{n}\}\\  %\sim\in \{<,\leq, =, \geq, >\}
	\mathcal{C}:= \{C^{S_1}_{\overrightarrow{n}},  \dots, C^{S_{n}}_{\overrightarrow{n}}\}\\  %\sim\in \{<,\leq, =, \geq, >\}
	\phi^{\mathcal{A}}_{C^{S_{i}}_{j}}:= a^{\mathcal{A}}_{C^{S_{i}}_{j}}\mid \neg \phi^{\mathcal{A}}_{i,j}\mid \phi^{\mathcal{A}}_{i,j}\rightarrow \phi^{\mathcal{A}}_{k,l}\mid \phi^{\mathcal{A}}_{i,j}\wedge \phi^{\mathcal{A}}_{k,l}\\
	\qquad \qquad \mid \phi^{\mathcal{A}}_{i,j}\vee \phi^{\mathcal{A}}_{k,l} \mid \bot \mid Read(\phi^{\mathcal{A}}_{C^{S_{i}}_{j}})\mid\\ 
	\qquad \qquad Write(\phi^{\mathcal{A}}_{C^{S_{i}}_{j}})\mid Trust(\phi^{\mathcal{A}}_{C^{S_{i}}_{j}})\\
%	RES:= \mathcal{M}^{\mathcal{A}}\mid mode\mid \neg RES\\
	\Gamma^{\mathcal{A}}:= 
	%\{\{\phi^{\mathcal{A}}_{i,j}, \dots, \phi^{\mathcal{A}}_{i,k}\}, \dots, \{\phi^{\mathcal{A}}_{j,l}, \dots, \phi^{\mathcal{A}}_{j,m}\}\}
	\phi^{\mathcal{A}}_{i,j} \mid \phi^{\mathcal{A}}_{i,j} < \phi^{\mathcal{A}}_{k,l} \mid \Gamma^{\mathcal{A}}; \phi^{\mathcal{A}}_{i,j}
	%\{ \phi^{S}_{i}< \dots < \phi^{S}_{n}\}. %\Gamma^{**}:=\{\psi^{**}_{1}, \dots, \psi^{**}_{n}\}\\
	
	\end{array}
	\end{displaymath}
\end{definition}
%

%\subsection{Services, Messaging and Protocols}
$\mathcal{A}$ is the set of agents issuing messages and including vehicles $\mathcal{V}$ and roadside units (RSUs) $\mathcal{R}$. Below we will focus in particular on V2V communication, without loss of generality. $\mathcal{S}$ denotes a set of services. $\mathcal{C}$ denotes a set of service characteristics, with each element $C^{S_i}_{\overrightarrow{n}}$ denoting the set of $n$ characteristics of service $S_{i}$. We assume, here and throughout, that characteristics $C^{S_i}_{\overrightarrow{n}}$ of services for each service $S_i$ is associated with an order $\leq$, so are given as \emph{posets}, and the ordering $\leq$ is used to order messages below in Definition \ref{def:dependencypackages}. (Note that for two characteristics from $C^{S_i}_{\overrightarrow{n}}$ and $C^{S_j}_{\overrightarrow{n}}$ respectively with $i\neq j$, there is no order between  these two characteristics.)


Messages are boolean formulae, closed under connectives and including $\bot$ to express conflicts. Messages are signed by agents generating them and by service and characteristic identifiers: $\phi^{v_{i}}_{C^{S_{k}}_{j}}$ expresses a  message $\phi$ about characteristic $C_j$ of service $S_k$  generated by vehicle $v_{i}$.  To simplify, we often abbreviate this notation as $\phi^{v_{i}}_{k,j}$.  When required, we will refer to a \textit{set of messages} about service $S_{k}$ and characteristic $C_{j}$ from vehicle $v_{i}$ as $\mathcal{M}^{v_{i}}_{S_{k}, C_{j}}$; this notation can be further generalised to a whole set of vehicles $\{v_{i}, \dots, v_{k}\} \subseteq  \mathcal{A}$. 
%
% $mode$ is a variable for reading, writing and trusting messages, closed under negation. 
A profile for vehicle $v_{i}$, denoted as $\Gamma^{v_{i}}$ is the current list of all messages collected by $v_{i}$ from available sensors, other agents and networks. For the present purposes, information from networks will be indexed at their first receiving vehicle, so as not to add networks as separate agents. For example, a vehicle profile $\Gamma^{v_{i}}$ might validate a message $\phi_{j,k}$ about service $S_{j}=\mathtt{weather}$ and characteristic $C_{k}=\mathtt{temperature}$ stating $\phi= (temp \geq 5^\circ C)$.


\begin{definition}[Formulae]
A formula $\Gamma^{v_l} \vdash_{\mathtt{s}} \phi^{v_{j}}_{i,k}$ states that a message $\phi$ about service $i$ and characteristic $k$ signed from agent $v_{j}$  is validly accessed at step $\mathtt{s}\geq 0$ under the profile of agent $v_{l}$.
\end{definition}

\begin{definition}[Validity]
A formula $\vdash_{\mathtt{s}} \phi^{v_{j}}_{i,k}$ says that a message $\phi$  about service $i$ and characteristic $k$ signed from vehicle $v_{j}$ holds for \textit{any} vehicle's profile at step $\mathtt{s}$.
\end{definition}

Messages satisfy a ranking based on characteristics:

\begin{definition}\label{def:dependencypackages}
We define an order $<$ between messages such that $\phi^{v_{j}}_{i,k}<\phi^{v_{j}}_{i,l}$ holds if $C^{S_{i}}_{k}\leq C^{S_{i}}_{l}$ for a vehicle $v_{j}$.
%
%says that for agent $v_{j}$ the characteristic $k$ of service $i$ is more important or essential than characteristic $l$ for the same service; then $\phi^{v_{j}}_{i,k} \vdash_{\mathtt{s}}  \phi^{v_{j}}_{i,l}$.
\end{definition}

The order relation $\leq$ between service characteristics induces therefore validity under profile: if a characteristic $k$ is essential to another characteristic $l$ with respect to a service $i$ for a vehicle $v_{j}$, then $v_i$ will be required to obtain a value for  $i$ in order to validly access a value for $l$. 

%This order relation between service characteristics can then be lifted at the level of services. A partial order relation $\leq$ over $\mathcal{S}\times \mathcal{S}$ intuitively expresses that an order is satisfied across services: 
%$S\leq S'$ means that repository $S$ contains a package that satisfies a dependency for a package in $S'$
%
%\begin{definition}\label{def:orderrepos}
%	$S_{i}\leq S_{j}$  iff $\exists \phi^{v_{j}}_{i,k},\phi^{v_{j}}_{j,l}\ s.t.\ \phi^{v_{j}}_{i,k}<\phi^{v_{j}}_{j,l}$.
%\end{definition}
%By the first clause in Definition \ref{def:orderrepos}, 
%$S_{i}< S_{j}$ means therefore that a service $S_{i}$ is essential for a service $S_{j}$ if there is a message concerning a characteristic of the latter which requires a message about a characteristic of the former. 
%The condition for $\leq$ is satisfied by a dependency relation $\phi_{i}^{A}<\phi_{j}^{A}$ between two packages in the same repository $A$. 
%The partial order allows for branching in the hierarchy, so that e.g. $S_{i}<S_{j}<S_{k}$ and $S_{i}<S_{j}<S_{l}$,  i.e. packages in $S_{k},S_{l}$  both require service $S_{j}$ and transitively $S_{i}$, but they have no requirement from each other. 
%By the second clause in Definition \ref{def:orderrepos}, our order relation abstracts from the issue of reciprocal dependencies. As noted in~\cite{DBLP:journals/eceasst/Boender11}, two packages that mutually depend on each other will either be installed together, or not installed at all. They can therefore be considered as a single package for dependency resolution purposes. 


A valid vehicle profile meets all the requirements and conflicts clauses of all service messages that the user collects.
%Context $\Gamma^{A}$ formalises a list of formulae describing an installation profile with packages from repository $A$, ordered by the dependency relation holding between relevant packages in $A$. 
%An installation profile can be extended by packages obtained by the same repository, denoted by $\Gamma^{A}, \phi^{A}$. 
Rules from Figure \ref{fig:system0} define vehicle's profile construction from service messages requirements. By Empty Profile, a user profile can be empty (base case); by Message Insertion, the elements in an installation profile are messages; by Requirement Insertion, a profile can be extended by satisfied service requirements; by Profile Extension, if a message holds in an empty profile, it can be added to an existing profile. In this syntax, the construction of two vehicles profiles $\Gamma^{v_{i}}; \Gamma^{v_{j}}: profile$ will typically denote the existence of an active communication channel between vehicles $v_{i}, v_{j}$.

\begin{figure*}[h]
	\begin{prooftree}
		\AxiomC{}
		\RightLabel{Empty Profile}
		\UnaryInfC{$\{\}: profile$}
		\DisplayProof
		\qquad
		\AxiomC{$\phi^{v_{j}}_{i,k} \!:\!\mathcal{M}^{v_{j}}$}
		\RightLabel{Message Insertion}
		\UnaryInfC{$ \phi^{v_{j}}_{i,k} \!:\!profile$}
	\end{prooftree}
	
	
	\begin{prooftree}
	   	\AxiomC{$\Gamma^{v_{j}}, \phi^{v_{j}}_{i,k}: profile$}
		\AxiomC{$\Gamma^{v_{j}}, \phi^{v_{j}}_{i,k}\vdash_{\mathtt{s}}  \psi^{v_{k}}_{i,l}$}
		\RightLabel{Requirement Insertion}
		\BinaryInfC{$\Gamma^{v_{j}}, \phi^{v_{j}}_{i,k}< \psi^{v_{k}}_{i,l} \!:\!profile$}
	\end{prooftree}


\begin{prooftree}
			\AxiomC{$\Gamma^{v_{i}}: profile$}
			\AxiomC{$\vdash_{\mathtt{s}}  \psi^{v_{k}}_{j,l}$}
			\RightLabel{Profile Extension}
			\BinaryInfC{$\Gamma^{v_{i}};\psi^{v_{k}}_{j,l} \!:\!profile$}
		\end{prooftree}
	
	
	\caption{The System \texttt{(un)SecureND}: Profile Construction Rules}\label{fig:system0}
\end{figure*}




\subsection{Rules for message construction}

The operational rules in Figure \ref{fig:system1} formulate compositionality of messages. The rule $Atom$ establishes that a vehicle and a communication channel between vehicles can qualify a message as valid if all its requirements are satisfied. $\bot$ expresses that contradictory messages imply access to their negation. $\wedge$-I allows to compose message originating from different vehicles; by $\wedge$-E, decomposition is valid for the channel obtained by the vehicles from which the messages originate. $\vee$-I says that a channel of two vehicles profiles can access any message produced from each of the composing vehicle profiles; by the elimination $\vee$-E, each message consistently inferred by each individual vehicle profile can also be executed under the channel between the profiles of the two vehicles. $\rightarrow$-Introduction expresses inference of a message from a channel as inference between messages (Deduction Theorem); its elimination $\rightarrow$-E allows to recover such inference as profile extension (Modus Ponens).

\begin{figure*}[t]
	\begin{prooftree}
		\AxiomC{$\Gamma^{v_{i}};\Gamma^{v_{j}}: profile$}
		\RightLabel{Atom, for any $\phi^{v_{j}}_{i,l}\in \Gamma^{v_{j}}$}
		\UnaryInfC{$\Gamma^{v_{i}};\Gamma^{v_{j}}\vdash_{\mathtt{s}}  \phi^{v_{j}}_{i,l}$}
		%\end{prooftree}
		\DisplayProof
		\qquad
		%
		%\begin{prooftree}
		\AxiomC{$\Gamma^{v_{i}} \vdash_{\mathtt{s}}  \phi^{v_{i}}_{i,j} \rightarrow \bot$}
		\RightLabel{$\bot$}
		\UnaryInfC{$\Gamma^{v_{i}}\vdash_{\mathtt{s+1}}  \neg \phi^{v_{i}}_{i,j}$}
	\end{prooftree}
	
	\begin{prooftree}
		\AxiomC{$\Gamma^{v_{i}} \vdash_{\mathtt{s}}  \phi^{v_{i}}_{i,l}$}
		\AxiomC{$\Gamma^{v_{j}} \vdash_{\mathtt{s'}}  \psi^{v_{j}}_{i,m}$}
		\RightLabel{$\wedge$-I}
		\BinaryInfC{$\Gamma^{v_{i}}; \Gamma^{v_{j}}  \vdash_{\mathtt{max(s,s')+1}}  \phi^{v_{i}}_{i,l} \wedge \psi^{v_{j}}_{i,m}$}
		\DisplayProof
		\qquad
		\AxiomC{$\Gamma^{v_{i}}; \Gamma^{v_{j}}  \vdash_{\mathtt{s}}  \phi^{v_{i}}_{i,l} \wedge \psi^{v_{j}}_{i,m}$}
		\RightLabel{$\wedge$-E}
		\UnaryInfC{$\Gamma^{v_{i}}; \Gamma^{v_{j}}  \vdash_{\mathtt{s+1}}  \phi/\psi^{v_{i/j}}_{i,l/m}$}
	\end{prooftree}
	
	\begin{prooftree}
		\AxiomC{$\Gamma^{v_{i}}; \Gamma^{v_{j}}  \vdash_{\mathtt{s}}  \phi^{v_{i/j}}_{i,l}$}
		\RightLabel{$\vee$-I}
		\UnaryInfC{$\Gamma^{v_{i}}; \Gamma^{v_{j}}  \vdash_{\mathtt{s+1}} \phi^{v_{i/j}}_{i,l} \vee \psi^{v_{i/j}}_{i,m}$}
		\DisplayProof
		\quad
		\AxiomC{$\Gamma^{v_{i}}; \Gamma^{v_{j}}  \vdash_{\mathtt{s}}  \phi^{v_{i/j}}_{i,l} \vee \psi^{v_{i/j}}_{i,m}$}
		\AxiomC{$\phi/\psi^{v_{i/j}}_{i,l/m} \vdash_{\mathtt{s'}}  \xi^{v_{i/j}}_{k,n}$}
		\RightLabel{$\vee$-E}
		\BinaryInfC{$\Gamma^{v_{i}}; \Gamma^{v_{j}}  \vdash_{\mathtt{max(s,s')+1}}  \xi^{v_{i/j}}_{k,n}$}
	\end{prooftree}
	
	
	%	with $I\in \{A,B\}, i \in\{1,2\}$ in the above rules.
	
	\begin{prooftree}
		\AxiomC{$\Gamma^{v_{i}};\phi^{v_{i}}_{i,l}\vdash_{\mathtt{s}}  \psi^{v_{j}}_{i,m}$}
		\RightLabel{$\rightarrow$-I}
		\UnaryInfC{$\Gamma^{v_{i}} \vdash_{\mathtt{s+1}}  \phi^{v_{i}}_{i,l} \rightarrow \psi^{v_{j}}_{i,m}$}
		\DisplayProof
		\qquad
		\AxiomC{$\Gamma^{v_{i}} \vdash_{\mathtt{s}}  \phi^{v_{i}}_{i,l} \rightarrow \psi^{v_{j}}_{i,m}$}
		\AxiomC{$\Gamma^{v_{i}} \vdash_{\mathtt{s'}}  \phi^{v_{i}}_{i,l}  $}
		\RightLabel{$\rightarrow$-E}
		\BinaryInfC{$\Gamma^{v_{i}};\phi^{v_{i}}_{i,l}\vdash_{\mathtt{max(s,s')+1}}  \psi^{v_{j}}_{i,m}$}
	\end{prooftree}
	
	
	
	\caption{The System \texttt{(un)SecureND}: Operational Rules}\label{fig:system1}
\end{figure*}



\subsection{Access Rules}

In Figure \ref{fig:system2} we present the access rules on messages. These allow a vehicle to act on messages from a distinct vehicle. $\neg$-distribution expresses profile consistency: if a vehicle profile does not allow inferring a message $\phi_{i,j}$, then it allows inferring any other message whose requirements do not include $\phi_{i,j}$. $\mathit{read}$ says that from any consistent vehicle profile a message can be read provided its requirements are satisfied (if any). $\mathit{trust}$ works as an elimination rule for $read$: it says that if a message is received by a vehicle and it preserves its profile consistency, then it can be trusted. $\mathit{write}$ works as an elimination rule for $trust$: it says that a message readable and trustable by a vehicle can be broadcast. $\mathit{exec}$ says that every message consistently received by a vehicle is valid in it. The rule MTrust-I says that currently held message conflicting with a newly arrived message are mistrusted, i.e., removed from the current vehicle profile until none of its consequences are included; the corresponding MTrust-E elimination allows to trust any message consistent with the conflict resolution by removal of the mistrusted message in the vehicle profile, including any required dependency: this is expressed by the side condition that requires checking with any other vehicle with higher reputation than the sender of the original message. The side condition can be modified at will, e.g., to design a protocol that will restore previous information if a sufficient number of other vehicles with higher reputation support it. \textit{mistrust} is a flag for facilitating removal of messages present in the vehicle profile conflicting in view of incoming new information. 




%\Gamma^{v_{i}};\Gamma^{v_{j}}\vdash_{\mathtt{s}}  \psi^{v_{j}}_{i,l}
\begin{figure*}[t]
	\centering
	\begin{prooftree}
		\AxiomC{$\Gamma^{v_{i}}\vdash_{\mathtt{s}}  \neg \mathcal{O}(\psi^{v_{j}}_{i,l})$}
		\RightLabel{$\mathcal{O}\in\{Read, Turst, Write\}, \neg$-distribution}
		\UnaryInfC{$\Gamma^{v_{i}}\vdash_{\mathtt{s+1}}  \mathcal{O}(\neg \psi^{v_{j}}_{i,l})$}
		\DisplayProof
		\qquad
		%
		\AxiomC{}
		\RightLabel{$\mathit{read}$}
		\UnaryInfC{$\Gamma^{v_{i}}  \vdash_{\mathtt{s}}  Read(\psi^{v_{j}}_{i,l})$}
		\end{prooftree}
		
       \begin{prooftree}
		\AxiomC{$\Gamma^{v_{i}}  \vdash_{\mathtt{s}}  Read(\psi^{v_{j}}_{i,l})$}
		\AxiomC{$\Gamma^{v_{i}};\psi^{v_{j}}_{i,l} : profile$}
		\RightLabel{$\mathit{trust}$}
		\BinaryInfC{$\Gamma^{v_{i}} \vdash_{\mathtt{s+1}}   Trust(\psi^{v_{j}}_{i,l} ) $}
	\end{prooftree}
	
	\begin{prooftree}
		\AxiomC{$\Gamma^{v_{i}}  \vdash_{\mathtt{s}}  Read(\psi^{v_{j}}_{i,l})$}
		\AxiomC{$\Gamma^{v_{i}} \vdash_{\mathtt{s'}}   Trust(\psi^{v_{j}}_{i,l} ) $}
		\RightLabel{$\mathit{write}$}
		\BinaryInfC{$\Gamma^{v_{i}} \vdash_{\mathtt{s'+1}}   Write(\psi^{v_{j}}_{i,l} ) $}
		\DisplayProof
		\qquad
		\AxiomC{$\Gamma^{v_{i}} \vdash_{\mathtt{s}}   Write(\psi^{v_{j}}_{i,l} ) $}
		\RightLabel{$\mathit{exec}$}
		\UnaryInfC{$\Gamma^{v_{i}} \vdash_{\mathtt{s+1}}   \psi^{v_{j}}_{i,l}  $}
	\end{prooftree}
	
%	\begin{prooftree}
%%		\AxiomC{$\Gamma^{A}\vdash_{\mathtt{s}}   wf$}
%		\AxiomC{$\Gamma^{v_{i}}  \vdash_{\mathtt{s}}  Read(\psi^{v_{j}}_{i,l})\rightarrow \bot $}
%		\RightLabel{DTrust-I}
%		\UnaryInfC{$\Gamma^{v_{i}} \vdash_{\mathtt{s}}   \neg Trust(\psi^{v_{j}}_{i,l} ) $}
%		\end{prooftree}
%		%\DisplayProof
%		%\qquad	
%		
%		\begin{prooftree}
%		\AxiomC{$\Gamma^{v_{i}} \vdash_{\mathtt{s}}   \neg Trust(\psi^{v_{j}}_{i,l} ) $}
%		\AxiomC{$\Gamma^{v_{i}} \vdash_{\mathtt{s}}   \neg Trust(\psi^{v_{j}}_{i,l}) \rightarrow \xi^{v_{k}}_{i,m}$}
%		\RightLabel{DTrust-E}
%		\BinaryInfC{$\Gamma^{v_{i}} \vdash_{\mathtt{s}}   Write(\xi^{v_{k}}_{i,m}) $}
%	\end{prooftree}
%	
	\begin{prooftree}
		\AxiomC{$\Gamma^{v_{i}} \vdash_{\mathtt{s}}   Read(\psi^{v_{j}}_{i,l})\rightarrow \bot $}
		\AxiomC{$\Gamma^{v_{i}}\setminus \{\neg\psi^{v_{i}}_{i,l}\} : profile$}
%		%\vdash_{\mathtt{s}}   wf, $\forall \phi^{A}_{j}\vdash_{\mathtt{s}}  Read(\psi^{B}_{i})\rightarrow \bot $}
		\RightLabel{MTrust-I}
		\BinaryInfC{$\Gamma^{v_{i}}\setminus \{\neg\psi^{v_{i}}_{i,l}\} \vdash_{\mathtt{s+1}}   \neg Trust(\neg \psi^{v_{i}}_{i,l})$}
	\end{prooftree}
%	
%
%	
	\begin{prooftree}
		\AxiomC{$\Gamma^{v_{i}}\setminus \{\neg\psi^{v_{i}}_{i,l}\} \vdash_{\mathtt{s}}   \neg Trust(\neg \psi^{v_{i}}_{i,l})$}
		\AxiomC{$\Gamma^{v_{k}}; \psi^{v_{j}}_{i,j} : profile$}
		\RightLabel{MTrust-E, $\forall v_{k}\prec v_{j}$}
		\BinaryInfC{$\Gamma^{v_{i}}\setminus \{\neg\psi^{v_{i}}_{i,l}\};\Gamma^{v_{k}}\vdash_{\mathtt{s+1}}  Trust(\psi^{v_{j}}_{i,l})$}
	\end{prooftree}
	\caption{The System \texttt{(un)SecureND}: Access Rules}\label{fig:system2}
\end{figure*}

%

\subsection{Structural Rules}

Structural rules hold with restrictions for $\mathtt{(un)SecureND}$, see Figure \ref{fig:system3}. As a result, the system qualifies as substructural, see for instance \cite{restall}. Weakening is constrained by an instance of $trust$: it says that valid information is preserved under a vehicle's profile extension, assuming the latter is provably consistent. Contraction is constrained by preservation of ordering: it says that removing identical messages from  a vehicle's profile is admissible, with the constraint that the copy from the vehicle with higher reputation is preserved. Exchange is constrained by dependency: it says that reorder of messages is admissible if there is no involved dependency between them. Finally, the Cut rule expresses validity under a vehicle's profile extension: if a message $\phi_{i,j}$ is validly for vehicle $v_{i}$ and after messaging it to $v_{j}$ the latter can infer $\phi_{i,k}$, then $v_{i}$ can infer $\phi_{i,k}$ by setting a message protocol with $v_{j}$.


\begin{figure*}[t]
	\begin{prooftree}
		\AxiomC{$\Gamma^{v_{i}}\vdash_{\mathtt{s}}  \phi^{v_{i}}_{i,j}$}
		\AxiomC{$\Gamma^{v_{i}}\vdash_{\mathtt{s'}}  Trust(\phi^{v_{j}}_{j,k})$}
		\RightLabel{Weakening}  
		\BinaryInfC{$\Gamma^{v_{i}}; \phi^{v_{j}}_{j,k}\vdash_{\mathtt{max(s,s'}+1)}  \phi^{v_{i}}_{i,j}$}
%	\end{prooftree}
\DisplayProof
%	
%		\begin{prooftree}
		\AxiomC{$\Gamma^{v_{i}}; \phi^{v_{j}}_{j,k}; \phi^{v_{k}}_{j,k}\vdash_{\mathtt{s}}  \psi^{v_{i}}_{i,j}$}
		\AxiomC{$v_{j} \prec v_{k}$}
		\RightLabel{Contraction}
		\BinaryInfC{$\Gamma^{v_{i}}; \phi^{v_{j}}_{j,k}\vdash_{\mathtt{s+1}}  \psi^{v_{i}}_{i,j}$}
	\end{prooftree}

	
	\begin{prooftree}
		\AxiomC{$\Gamma^{v_{i}}; \phi^{v_{i}}_{i,j}; \phi^{v_{i}}_{i,k}\vdash_{\mathtt{s}}  \psi^{v_{i}}_{i,j}$}
		\AxiomC{$\phi^{v_{i}}_{i,j}\nless \phi^{v_{i}}_{i,k}$}
		\RightLabel{Profile Exchange}
		\BinaryInfC{$\Gamma^{v_{i}}; \phi^{v_{i}}_{i,k}; \phi^{v_{i}}_{i,j}\vdash_{\mathtt{s+1}}  \psi^{v_{i}}_{i,j}$}
	\end{prooftree}
%\DisplayProof

	
	\begin{prooftree}
		\AxiomC{$\Gamma^{v_{i}} \vdash_{\mathtt{s}}  \phi^{v_{i}}_{i,j}$}
		\AxiomC{$\Gamma^{v_{j}}, \phi^{v_{i}}_{i,j}\vdash_{\mathtt{s'}}   \phi^{v_{j}}_{i,k}$ }
		\RightLabel{Cut}
		\BinaryInfC{$\Gamma^{v_{i}}; \Gamma^{v_{j}}\vdash_{\mathtt{max(s,s')+1}}   \phi^{v_{j}}_{i,k}$}
	\end{prooftree}
	\caption{The System \texttt{(un)SecureND}: Structural Rules}\label{fig:system3}
\end{figure*}

%The cut rule justifies the following result:

\begin{theorem}[Normalization]
Any message $\phi_{i,k}$ valid for a channel $v_{i},v_{j}$ and obtained by an occurrence $c$ of the $Cut$ rule can be validated without $c$ using only trust.	
\end{theorem}

\begin{proof}
	By induction on the derivation $D$ which is the redex of the cut-elimination. Assuming $c$ is the only Cut rule and it is the last inference rule of the redex, the derivation $D^\prime$ which is the contractum of the cut-elimination contains a descendent of the cut obtained by an instance of Weakening under trust. Because the formula obtained by the cut is, by hypothesis, derivable from the weaker protocol, it will also be derivable from the weaker and the stronger protocol together. When $c$ is not the last inference rule of the redex, then the descendent of the cut will admit all similar Weakenings preserving the one occurring in the cut; those imports by Weakening will occur also in the contractum of the cut rule and can be traced back up to the one formulation of the import that occurs in the cut rule.
\end{proof}

Normalization justifies a safety property of our trust and reputation model over transitive transmissions: for each vehicle $v_{i}, v_{j}, v_{k}$, if $v_{k}$ holds information $\phi_{i,j}$ and this information is passed to $v_{j}$, then every valid message derived from $\phi_{i,j}$ by $v_{k}$ can be inferred by $v_{j}$ assuming the consistency (by trust) of its profile with that of $v_{k}$; similarly now, $v_{j}$ can pass $\phi_{i,j}$ to $v_{i}$, and the latter can infer from there, assuming its profile is consistent with those of $v_{j}, v_{k}$.



\section{Opportunistic Forwarding}\label{sec:opportunistic}

%\subsection{Handshaking Protocol}


In this section we present the algorithm and exemplify derivations for handshaking and opportunistic message forwarding protocols. The algorithm consists of two parts: it first implements the recipient selection of a communication by reputation and then the message forwarding if consistency is guaranteed by trust. The pseudo-code of the full protocol with handshaking and opportunistic forwarding is formulated in Figure \ref{fig:routine1}. Here we use protocol operations named after the relevant $\mathtt{SecureND}$ rules, as well as symbols for vehicles, services and characteristics.


\begin{figure}[t]
	\lstset{language=Java,
	%	numbers=left,
		basicstyle=\scriptsize,
		mathescape}
%	
	\begin{lstlisting}[frame=single]  % Start your code-block
	
PROCEDURE $\mathtt{Opportunistic Forwarding}(v_{i},v_{j})$
	
	IF $v_{i}$  $Write(HELLO)$ 
	    THEN forall $[v_{k} \in \mathcal{A}\mid v_{k}$ $Write(HELLO)]$, 
					SELECT $min(v_{k}, \prec)$
					DO Handshaking$(v_{i},v_{k})$
	ENDIF
			
	IF Handshaking($v_{i},v_{k}$)
		THEN $v_{i}$ $Write(\phi_{i,k})$ AND $v_{k}$ $Read(\phi_{i,k})$ 
					IF $v_{k}$ $Trust(\phi_{i,k})$
					THEN $v_{k}$ $Write(\phi_{i,k})$
					ELSE  $v_{k}$  $\neg Trust (\phi_{i,k})$
					ENDIFELSE
					IF forall $v_{i}\prec v_{k}$, $v_{i}$ $Trust(\phi_{i,k})$
						THEN $v_{k}$ $Trust(\phi_{i,k})$
					ELSE $v_{k}$ $\neg Trust(\phi_{i,k})$
					ENDIFELSE
	ENDIF
				
ENDPROCEDURE
	\end{lstlisting}
	\caption{Algorithm Opportunistic Forwarding}\label{fig:routine1}
\end{figure}



In Figure \ref{fig:handshake} we present the \texttt{SecureND} translation of the handshaking protocol. Here Service 1 identifies the set of messages for this protocol. By Hello Message, a user $v_{i}$ with a well-defined profile with a `hello' message in its recognition service sends the message to the network;   a user $v_{k}$ reading the message and assuming it preserves consistency (e.g. there is no instruction in its profile to ingore messages from $v_{i}$), accepts it and forwards it further, including a `hello' back to $v_{i}$.


\begin{figure*}
	\begin{prooftree}
		\AxiomC{$\Gamma^{v_{i}}:profile$}
		\AxiomC{$\Gamma^{v_{i}}\vdash_{\mathtt{1}}  hello^{v_{i}}_{1,1}$}
		\RightLabel{Hello Message}
		\BinaryInfC{$\Gamma^{v_{i}}\vdash_{\mathtt{2}}  Write(hello^{v_{i}}_{1,1})$}
		\end{prooftree}

	\begin{prooftree}
		\AxiomC{$\Gamma^{v_{i}}\vdash_{\mathtt{1}}  Write(hello^{v_{i}}_{1,1})$}
		\AxiomC{$\Gamma^{v_{k}}\vdash_{\mathtt{2}}  Read(hello^{v_{i}}_{1,1})$}
				
		\AxiomC{$\Gamma^{v_{k}}; hello^{v_{i}}_{1,1}:profile$}
		\RightLabel{Response Message}
		\TrinaryInfC{$\Gamma^{v_{k}}; hello^{v_{i}}_{1,1}\vdash_{\mathtt{3}}  Write(hello^{v_{k}}_{1,1})$}
		\end{prooftree}


	\caption{The Handshaking Protocol}\label{fig:handshake}
\end{figure*}


%\subsection{Recipient Selection Protocol}

In Figure \ref{fig:selection}, we present an example derivation of the recipient selection protocol. Here the idea is as follows: after $v_{i}$ broadcasts a `hello' message, both $v_{k}, v_{j}$ receive and accept the message; at this stage a recipient is selected on the basis of the reputation order between $v_{k}$ and $v_{j}$, so that a new profile is built out of $v_{i}$ and the higher of the two recipients, thus modelling a communication channel.


\begin{figure*}
	\begin{prooftree}
		\AxiomC{$\Gamma^{v_{j,\dots, n}}; hello^{v_{i}}_{1,1}\vdash_{\mathtt{1}}  Write(hello^{v_{j,\dots, n}}_{1,1})$}
%		\AxiomC{$\Gamma^{v_{j}}; hello^{v_{i}}_{1,1}\vdash_{\mathtt{2}}  Write(hello^{v_{j}}_{1,1})$}
		\AxiomC{$v_{l}\in min(v_{i, \dots, n}, \prec)$}
		\RightLabel{Recipient Selection}
		\BinaryInfC{$\Gamma^{v_{i}}; \Gamma^{v_{l}}:profile$}
		\end{prooftree}

	\caption{The Handshaking Protocol}\label{fig:selection}
\end{figure*}


%\subsection{Message Passing Protocol}

In Figure \ref{fig:mp}, we present an example derivation modelling a message passing protocol (without mistrust). Here Service 2 is some service of any kind. By the first premise in MP, the Handshaking Protocol is guaranteed terminating, including the Recipient Selection protocol if required; $v_{k}$ then reads a message issued by $v_{i}$, checks for validity in its own profile through an application of $trust$, and if this check is passed the message is forwarded.



\begin{figure*}
	\begin{prooftree}
		\AxiomC{$\Gamma^{v_{i}}; \Gamma^{v_{k}}:profile$}
		\AxiomC{$\Gamma^{v_{i}}\vdash_{\mathtt{1}}  Write(m^{v_{i}}_{2,1})$}
		\RightLabel{MP}
		\BinaryInfC{$\Gamma^{v_{k}} \vdash_{\mathtt{2}}  Read(m^{v_{i}}_{2,1})$}
		\AxiomC{$\Gamma^{v_{k}}; m^{v_{i}}_{2,1}: profile$}
		\BinaryInfC{$\Gamma^{v_{k}} \vdash_{\mathtt{3}}  Trust(m^{v_{i}}_{2,1})$}
		\UnaryInfC{$\Gamma^{v_{k}} \vdash_{\mathtt{4}}  Write(m^{v_{i}}_{2,1})$}
		\end{prooftree}

	\caption{The Message Passing Protocol}\label{fig:mp}
\end{figure*}



\section{Reputation Model}\label{sec:reputation}

In this section we illustrate the definition of the order relation $\prec$ to formalise the reputation model across vehicles. Higher reputation is modelled by feedback aggregation. Our system integrates the elements of the main feedback 6-tuple function from \cite{glenford}. In particular, time is encoded directly by derivation steps; context is embedded by the user profile; service and characteristics are modelled by messages. To model the set of feedback that a given vehicle provides with respect to a given message related to a service and characteristic, we will have to collect all formulae following receiving a message:

\begin{definition}[Feedback Set]
The feedback set of vehicle $v_{j}$ for a message $\phi^{v_{i}}_{i,j}$, for all $v_{j}, v_{i} \in \mathcal{A}$ is the set of formulas $\psi^{v_{j}}_{i,k}$ such that they agree with $\phi^{v_{i}}_{i,j}$ for the service identifier $i$ and are obtained by a derivation construed by a $read$ rule followed by a $\rightarrow I$ rule, i.e.
%
\[
FS^{v_{j}}(\phi^{v_{i}}_{i,j})=\{ \psi^{v_{j}}_{i,k}\mid \Gamma^{v_{j}}
\vdash_{\mathtt{s}}  Read(\phi^{v_{i}}_{i,j})\rightarrow \psi^{v_{j}}_{i,k}  \}
\]
\end{definition}

By way of example, consider the following simple derivation, which induces $FS^{v_{k}}(m^{v_{i,j}}_{2,1})=\{m^{v_{k}}_{2,2}\}$: 

\begin{figure*}
\begin{footnotesize}

	\begin{prooftree}
		\AxiomC{$\Gamma^{v_{i}}; \Gamma^{v_{k}}:profile$}
		\AxiomC{$\Gamma^{v_{j}}; \Gamma^{v_{k}}:profile$}
		\BinaryInfC{$\Gamma^{v_{i}};\Gamma^{v_{j}}; \Gamma^{v_{k}}:profile$}
		
		\AxiomC{$\Gamma^{v_{k}}\vdash_{\mathtt{1}}  Write(m^{v_{i,j}}_{2,1})$}
%		\AxiomC{$\Gamma^{v_{k}}\vdash_{\mathtt{s}}  Write(m^{v_{j}}_{2,1})$}
		\BinaryInfC{$\Gamma^{v_{k}} \vdash_{\mathtt{2}}  Read(m^{v_{i,j}}_{2,1})$}
		\AxiomC{$\Gamma^{v_{k}}; m^{v_{i,j}}_{2,1}: profile$}
		\BinaryInfC{$\Gamma^{v_{k}} \vdash_{\mathtt{3}}  Trust(m^{v_{i,j}}_{2,1})$}
		\UnaryInfC{$\Gamma^{v_{k}} \vdash_{\mathtt{4}}  Write(m^{v_{i,j}}_{2,1})$}
		\AxiomC{$\Gamma^{v_{k}}; m^{v_{i,j}}_{2,1}\vdash_{\mathtt{5}}  m^{v_{k}}_{2,2}$}
        \BinaryInfC{$\Gamma^{v_{k}} \vdash_{\mathtt{6}}  m^{v_{i}}_{2,1} \rightarrow m^{v_{k}}_{2,2}$}
		\end{prooftree}
\end{footnotesize}
\caption{An Example Feedback Set}\label{fig:ask}
\vspace{-2mm}
\end{figure*}


Notice that, by construction, this set includes only feedback to received messages that are consistent with the current user's profile. 

\begin{definition}[Vehicle's Perception]
The perception of vehicle $v_{j}$ for a message $\phi^{v_{i}}_{i,j}$, for all $v_{j}, v_{i} \in \mathcal{A}$ is the sum of elements of the feedback set over that formula, weighted by the step of the derivation at which it is obtained: 

\[
AP^{v_{j}}(\phi^{v_{i}}_{i,j})=\sum_{FS^{v_{i}}(\phi^{v_{j}}_{i,k})}(\mathtt{s}(\psi^{v_{j}}_{i,k} \in FS^{v_{i}}(\phi^{v_{j}}_{i,k})))
%\mid \Gamma^{v_{j}}
%\vdash_{\mathtt{s}}  Read(\phi^{v_{i}}_{i,j})\rightarrow \psi^{v_{j}}_{i,k}  \}
\]

\end{definition}

Intuitively, the value of $\mathtt{s}$ at each step of each derivation leading to each formula in the feedback set of a vehicle to a given service and characteristic is summed up to provide a value that increases linearly to reflect a step value for a time function. The value of $AP^{v_{j}}(\phi^{v_{i}}_{i,j})$ will reflect the aggregation of all the feedback provided on each characteristics of a given service.  


We can now generalise to the set of all feedback on a characteristic for a given service, remembering that these are given in a preorder so that the position of the characteristic in that order is mapped into an integer:


\begin{definition}[Vehicle's Perception of Characteristic Set]
The perception of vehicle $v_{j}$ for a set of messages $\mathcal{M}^{\mathcal{A}}_{S_{i},C_{k}}$ from other vehicles about characteristic $C_{k}$ of service $S_{i}$ is the sum of elements of the feedback set over the messages received about that service characteristic, weighted by the steps of the derivation at which it is obtained and further by the value $\mathtt{r}(C_{k})$ of the rank of characteristic $k$: 
%
\begin{displaymath}
\begin{array}{l}
AP^{v_{j}}(\mathcal{M}^{\mathcal{A}}_{S_{i}, C_{k}})=\\
\sum_{FS^{v_{i}}(\phi^{v_{j}}_{i,k}\dots \phi^{v_{n}}_{i,k})}
(1-\mathtt{r}(C_{k})(\mathtt{s}(\psi^{v_{j}}_{i,k} \in FS^{v_{i}}(\phi^{v_{j}}_{i,k}\dots \phi^{v_{n}}_{i,k}))))
\end{array}
\end{displaymath}
\end{definition}


Using the vehicle's perception of characteristic set, we can define the order of reputation with respect to services and characteristics, which establishes a higher position for the vehicle whose perception on the characteristics set for that Service is greater.

\begin{definition}[Reputation]
$\forall v_{i}, v_{j}\in \mathcal{V}, S_{i}\in \mathcal{S}, v_{i}\prec v_{j} \leftrightarrow AP^{v_{i}}(\mathcal{M}^{\mathcal{A}}_{S_{i}, C_{k}})>AP^{v_{j}}(\mathcal{M}^{\mathcal{A}}_{S_{i}, C_{k}})$.
\end{definition}

\section{Conclusions}

In this paper we have formulated a proof-theory for trust and reputation in VANETs. Our language is modelled on the logic \texttt{(un)SecureND}, including an explicit $trust$ function on formulas to guarantee consistency check at each retrieval step (after a $read$ function), before forwarding is granted for a package (by a $write$ function). Forwarding is modelled in an opportunistic fashion, selecting receivers on the basis of their reputation ranking. Trust on forwarding also guarantees correctness on transitive transmissions. Moreover, reputation is used to implement the resolution protocol for restoring information after removing previously stored data. Several improvements for the algorithm are possible, including majority selection on opportunistic forwarding (instead of consensus) and separate ordering for vehicles and RSUs. Several improvements for the algorithm are possible, including majority selection on opportunistic forwarding (instead of consensus) and separate ordering for vehicles and RSUs.
% Different pairs of introduction/elimination rules determine the selection of one of tow resolution strategies: one flags a package external to the installation profile as distrusted and hence as not installable; the other identifies already installed packages to be removed. The selection takes care to identify and remove all required dependencies. We have illustrated the working protocol through an easy example. 
Validation of the system is obtained by implementation of the \texttt{(un)SecureND} calculus as a large inductive type in the Coq proof assistant. The development is available at \url{https://github.com/gprimiero/SecureNDC}. 
%It makes it possible to express and prove the lemmas and theorems from sections~\ref{sec:distrusted} and~\ref{sec:mistrusted}.
A characteristic of the logic \texttt{(un)SecureND} is its substructural nature, which in future work can be exploited to investigate cases of strengthened and limited resource redundancy for fault tolerance and source shuffling for security. Other applications of negative trust can be investigated to distinguish between malevolent and simply unsuccessful sources.

%
%\bibliographystyle{plain}
%\bibliography{unsecureNDC}

\begin{thebibliography}{10}
%\vspace{-2mm}
\bibitem{bouassida2011authentication}
Mohamed~Salah Bouassida.
\newblock Authentication vs. privacy within vehicular ad hoc networks.
\newblock {\em International Journal of Network Security}, 13(3):121--134,
  2011.

\bibitem{DBLP:journals/ijaisc/ChaurasiaTV15}
Brijesh~Kumar Chaurasia, Ranjeet~Singh Tomar, and Shekhar Verma.
\newblock {Using trust for lightweight communication in VANETs}.
\newblock {\em {IJAISC}}, 5(2):105--116, 2015.

\bibitem{DBLP:conf/um/FinnsonZTMC12}
John Finnson, Jie Zhang, Thomas~T. Tran, Umar~Farooq Minhas, and Robin Cohen.
\newblock {A Framework for Modeling Trustworthiness of Users in Mobile
  Vehicular Ad-Hoc Networks and Its Validation through Simulated Traffic Flow}.
\newblock In {\em {User Modeling, Adaptation, and Personalization - 20th
  International Conference, {UMAP} 2012.
  Proceedings}}, volume 7379 of {\em {Lecture Notes in Computer Science}},
  pages 76--87. Springer, 2012.

\bibitem{GomezMarmol:2012:TTR:2160992.2161100}
F{\'e}lix {G{\'o}mez M{\'a}rmol} and Gregorio {Mart{\'i}nez P{\'e}rez}.
\newblock {TRIP, a Trust and Reputation Infrastructure-based Proposal for
  Vehicular Ad Hoc Networks}.
\newblock {\em J. Netw. Comput. Appl.}, 35(3):934--941, May 2012.

\bibitem{tesla-cpn}
M.~H. Jahanian, F.~Amin, and A.~H. Jahangir.
\newblock Analysis of tesla protocol in vehicular ad hoc networks using timed
  colored petri nets.
\newblock In {\em 2015 6th International Conference on Information and
  Communication Systems (ICICS)}, pages 222--227, April 2015.

\bibitem{DBLP:conf/vtc/KonurF11}
Savas Konur and Michael Fisher.
\newblock {Formal Analysis of a {VANET} Congestion Control Protocol through
  Probabilistic Verification}.
\newblock In {\em {Proceedings of the 73rd {IEEE} Vehicular Technology
  Conference, {VTC} Spring 2011, 15-18 May 2011, Budapest, Hungary}}, pages
  1--5. {IEEE}, 2011.

\bibitem{Lo2009}
Nai-Wei Lo and Hsiao-Chien Tsai.
\newblock {A Reputation System for Traffic Safety Event on Vehicular Ad Hoc
  Networks}.
\newblock {\em EURASIP Journal on Wireless Communications and Networking},
  2009(1):125348, 2009.

\bibitem{5641621}
U.~F. Minhas, J.~Zhang, T.~Tran, and R.~Cohen.
\newblock {A Multifaceted Approach to Modeling Agent Trust for Effective
  Communication in the Application of Mobile Ad Hoc Vehicular Networks}.
\newblock {\em IEEE Transactions on Systems, Man, and Cybernetics, Part C
  (Applications and Reviews)}, 41(3):407--420, May 2011.

\bibitem{DBLP:conf/ifiptm/Primiero16}
Giuseppe Primiero.
\newblock {A Calculus for Distrust and Mistrust}.
\newblock In {\em {Trust Management {X} - 10th {IFIP} {WG}
  11.11 International Conference, {IFIPTM}, Proceedings}}, volume 473 of {\em {{IFIP} Advances in
  Information and Communication Technology}}, pages 183--190. Springer, 2016.

\bibitem{primiero_secureND}
Giuseppe Primiero and Franco Raimondi.
\newblock {A typed natural deduction calculus to reason about secure trust}.
\newblock In Ali Miri, Urs Hengartner, Nen{-}Fu Huang, Audun J{\o}sang, and
  Joaqu{\'i}n Garc{\'i}a{-}Alfaro, editors, {\em {2014 Twelfth Annual
  International Conference on Privacy, Security and Trust}}, pages 379--382. {IEEE}, 2014.

\bibitem{conf/infocom/RayaPGH08}
Maxim Raya, Panagiotis Papadimitratos, Virgil~D. Gligor, and Jean-Pierre
  Hubaux.
\newblock {On Data-Centric Trust Establishment in Ephemeral Ad Hoc Networks.}
\newblock In {\em {INFOCOM}}, pages 1238--1246. IEEE, 2008.

\bibitem{restall}
Greg Restall.
\newblock {\em {An Introduction to Substructural Logics}}.
\newblock Routledge, 2000.

\bibitem{Soleymani2015}
Seyed~Ahmad Soleymani, Abdul~Hanan Abdullah, Wan~Haslina Hassan,
  Mohammad~Hossein Anisi, Shidrokh Goudarzi, Mir~Ali {Rezazadeh Baee}, and
  Satria Mandala.
\newblock {Trust management in vehicular ad hoc network: a systematic review}.
\newblock {\em EURASIP Journal on Wireless Communications and Networking},
  2015(1):146, 2015.

\bibitem{glenford}
R~Vanni, L.M.S. Jaimes, G.~Mapp, and E.~Moreira.
\newblock {Ontology Driven Reputation Model for VANET}.
\newblock In {\em {AICT 2016, The Twelfth Advanced International Conference on
  Telecommunications }}, pages 14--19. IARIA, 2016.

\bibitem{Wei2012}
Yu-Chih Wei and Yi-Ming Chen.
\newblock {\em {Reliability and Efficiency Improvement for Trust Management
  Model in VANETs}}, pages 105--112.
\newblock Springer Netherlands, Dordrecht, 2012.

\bibitem{DBLP:conf/vtc/WexBHLD08}
Philipp Wex, Jochen Breuer, Albert Held, Tim Leinm{\"u}ller, and Luca
  Delgrossi.
\newblock {Trust Issues for Vehicular Ad Hoc Networks}.
\newblock In {\em {Proceedings of the 67th {IEEE} Vehicular Technology
  Conference, {VTC} Spring 2008, 11-14 May 2008, Singapore}}, pages 2800--2804.
  {IEEE}, 2008.

\end{thebibliography}

\end{document}

